% Notes and exercises from Introduction to Real Analysis by DePree and Swartz
% By John Peloquin
\documentclass[letterpaper,12pt]{article}
\usepackage{amsmath,amssymb,amsthm,enumitem,fourier}

\newcommand{\R}{\mathbb{R}}

% Theorems
\theoremstyle{plain}
\newtheorem*{thm}{Theorem}

\theoremstyle{definition}
\newtheorem*{exer}{Exercise}

\theoremstyle{remark}
\newtheorem*{rmk}{Remark}

% Meta
\title{Notes and exercises from\\\textit{Introduction to Real Analysis}}
\author{John Peloquin}
\date{}

\begin{document}
\maketitle

\section*{Introduction}
This document contains notes and exercises from~\cite{depree}.

\section*{Chapter~13}
We provide an alternative proof of the forward direction of Theorem~42.
\begin{thm} If \(f:[a,b]\to\R\) is integrable, then
\[\int_a^b f=\lim_{c\to a^+}\int_c^b f\]
\end{thm}
\begin{proof}
Note \(\int_c^b f\)~exists for all \(c\in[a,b]\) by Theorem~21. Now
\begin{align*}
\lim_{c\to a^+}\int_c^b f&=\lim_{c\to a^+}\Bigl[\,\int_a^b f-\int_a^c f\,\Bigr]&&\text{by Theorem~18}\\
	&=\int_a^b f-\lim_{c\to a^+}\int_a^c f&&\\
	&=\int_a^b f-\int_a^a f&&\text{by Theorem~27}\\
	&=\int_a^b f&&\qedhere
\end{align*}
\end{proof}
\begin{rmk}
We used the continuity of the indefinite integral (Theorem~27), which was proved using Henstock's Lemma (Lemma~25). The proof of the forward direction of Theorem~42 in the book unnecessarily repeats some ideas from the proof of Henstock's Lemma. See also the proof of the forward direction of Theorem~14.6 in the book.
\end{rmk}

% References
\begin{thebibliography}{0}
\bibitem{depree} DePree, John~D. and Charles W. Swartz. \textit{Introduction to Real Analysis.} Wiley, 1988.
\end{thebibliography}
\end{document}
