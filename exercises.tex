% Notes and exercises from Introduction to Real Analysis by DePree and Swartz
% By John Peloquin
\documentclass[letterpaper,12pt]{article}
\usepackage{amsmath,amssymb,amsthm,enumitem,fourier}

\newcommand{\Q}{\mathbb{Q}}
\newcommand{\R}{\mathbb{R}}
\newcommand{\Null}{\mathcal{N}}
\newcommand{\D}{\mathcal{D}}
\renewcommand{\L}{\mathcal{L}}
\newcommand{\BC}{\mathcal{BC}}
\newcommand{\C}{\mathcal{C}}
\newcommand{\F}{\mathcal{F}}
\renewcommand{\P}{\mathcal{P}}
\newcommand{\I}{\mathcal{I}}
\newcommand{\T}{\mathcal{T}}

\newcommand{\upto}{\uparrow}
\newcommand{\downto}{\downarrow}

\newcommand{\union}{\cup}
\newcommand{\sect}{\cap}
\newcommand{\bigunion}{\bigcup}
\newcommand{\bigsect}{\bigcap}
\newcommand{\fmin}{\wedge}
\newcommand{\fmax}{\vee}
\newcommand{\conv}{\ast}
\newcommand{\after}{\circ}

\newcommand{\Rint}{R\!\int}
\renewcommand{\d}[1]{\,d\!{#1}}
\newcommand{\dt}{\d{t}}
\newcommand{\du}{\d{u}}
\newcommand{\dx}{\d{x}}
\newcommand{\dy}{\d{y}}
\newcommand{\df}{d\!f}
\newcommand{\dg}{dg}
\renewcommand{\dh}{d\!h}

\DeclareMathOperator{\Lap}{\mathcal{L}}

\newcommand{\abs}[1]{|{#1}|}
\newcommand{\bigabs}[1]{\left|{#1}\right|}
\newcommand{\norm}[1]{\lVert{#1}\rVert}
\newcommand{\bignorm}[1]{\left\lVert{#1}\right\rVert}
\newcommand{\supnorm}[1]{\norm{#1}_{\infty}}
\newcommand{\pos}[1]{#1^{+}}
\renewcommand{\neg}[1]{#1^{-}}
\renewcommand{\vec}[1]{\boldsymbol{#1}}
\newcommand{\comp}[1]{#1^c}
\newcommand{\closure}[1]{\overline{#1}}

% Theorems
\theoremstyle{plain}
\newtheorem*{thm}{Theorem}
\newtheorem*{prop}{Proposition}

\theoremstyle{definition}
\newtheorem*{exmp}{Example}
\newtheorem*{exer}{Exercise}

\theoremstyle{remark}
\newtheorem*{rmk}{Remark}

% Meta
\title{Notes and exercises from\\\textit{Introduction to Real Analysis}}
\author{John Peloquin}
\date{}

\begin{document}
\maketitle

\section*{Introduction}
This document contains notes and exercises from~\cite{depree}.

\section*{Chapter~5}
We provide an alternative proof of Cauchy-Schwarz for~\(\R^n\) (Proposition~2(v)).
\begin{thm}[Cauchy-Schwarz]
If \(\vec{x},\vec{y}\in\R^n\), then \(\abs{\vec{x}\cdot\vec{y}}\le\norm{\vec{x}}\norm{\vec{y}}\).
\end{thm}
\begin{proof}
Let \(\vec{x}=(x_1,\ldots,x_n)\) and \(\vec{y}=(y_1,\ldots,y_n)\). We must prove
\[\bigabs{\sum_{k=1}^n x_ky_k}\le\sqrt{\sum_{k=1}^n x_k^2}\ \sqrt{\sum_{k=1}^n y_k^2}\tag{1}\]
Let \(B=2\sum_{k=1}^n x_ky_k\), \(A=\sum_{k=1}^n x_k^2\), and \(C=\sum_{k=1}^n y_k^2\). Then (1)~is equivalent to \((B/2)^2-AC\le 0\), which is equivalent to \(D=B^2-4AC\le 0\). We may assume \(A\ne 0\), so \(D\)~is just the discriminant of the quadratic polynomial \(P(x)=Ax^2+Bx+C\) and \(D\le 0\) if \(P\ge 0\). Now
\[Ax^2+Bx+C=\sum_{k=1}^n(xx_k)^2+\sum_{k=1}^n2(xx_k)y_k+\sum_{k=1}^ny_k^2=\sum_{k=1}^n(xx_k+y_k)^2\ge 0\qedhere\]
\end{proof}

\section*{Chapter~13}
We provide an alternative proof of Theorem~5.
\begin{thm}
If \(\gamma\)~is a gauge on \(I=[a,b]\), there exists a \(\gamma\)-fine tagged division of~\(I\).
\end{thm}
\begin{proof}
If not, bisect~\(I\). At least one of the two resulting subintervals must not have a \(\gamma\)-fine tagged division, lest we could combine \(\gamma\)-fine tagged divisions of both to obtain a \(\gamma\)-fine tagged division of~\(I\).

Continue bisecting recursively to obtain a descending chain \(I=I_0\supseteq\cdots\supseteq I_k\supseteq\cdots\) of subintervals with \(l(I_k)\to0\) such that \(I_k\)~does not have a \(\gamma\)-fine tagged division. Let \(x\in\bigsect_{k=0}^{\infty}I_k\). Then there exists \(I_k\subseteq\gamma(x)\), so \(\{(x,I_k)\}\)~is a \(\gamma\)-fine tagged division of~\(I_k\)---a contradiction.
\end{proof}

\begin{rmk}
In the proof of the second part of the fundamental theorem of calculus (Theorem~28), note that \(f-f(x)\) is absolutely integrable by Corollary~33.
\end{rmk}

\noindent We provide an alternative proof of the forward direction of Theorem~42.
\begin{thm} If \(f:[a,b]\to\R\) is integrable, then
\[\int_a^b f=\lim_{c\to a^+}\int_c^b f\]
\end{thm}
\begin{proof}
Note \(\int_c^b f\)~exists for all \(c\in[a,b]\) by Theorem~21. Now
\begin{align*}
\lim_{c\to a^+}\int_c^b f&=\lim_{c\to a^+}\left[\,\int_a^b f-\int_a^c f\,\right]&&\text{by Theorem~18}\\
	&=\int_a^b f-\lim_{c\to a^+}\int_a^c f&&\\
	&=\int_a^b f-\int_a^a f&&\text{by Theorem~27}\\
	&=\int_a^b f&&\qedhere
\end{align*}
\end{proof}
\begin{rmk}
We used the continuity of the indefinite integral (Theorem~27), which was proved using Henstock's Lemma (Lemma~25). The proof of the forward direction of Theorem~42 in the book unnecessarily repeats some ideas from the proof of Henstock's Lemma. See also the proof of the forward direction of Theorem~14.6 in the book.

Our proof is natural because the theorem itself expresses the continuity of the integral when viewed as a function of the interval of integration. See also the remarks on the monotone convergence theorem below.
\end{rmk}

\section*{Chapter~14}
\begin{exer}[12]
Let \(f:[a,\infty)\to\R\) be continuous and such that the indefinite integral \(F(x)=\int_a^x f\) is bounded. Let \(g:[a,\infty)\to\R\) be nonnegative, decreasing, and differentiable. Then \(\int_a^{\infty}fg\)~exists if either \(\lim_{x\to\infty}g(x)=0\) or \(\int_a^{\infty}f\)~exists.
\end{exer}
\begin{proof}
For \(b\in[a,\infty)\), \(fg\)~is continuous on~\([a,b]\), so \(\int_a^b fg\)~exists (Theorem~13.23). By the fundamental theorem of calculus (Theorem~13.28), \(F'=f\) on~\([a,b]\), so integration by parts (Proposition~13.17) yields
\[\int_a^b fg=\int_a^b F'g=F(b)g(b)-F(a)g(a)-\int_a^b Fg'=F(b)g(b)-\int_a^b Fg'\]
Now \(\lim_{b\to\infty}g(b)\) exists since \(g\)~is decreasing and bounded below by zero. If \(\lim_{b\to\infty}g(b)=0\), then boundedness of~\(F\) implies \(\lim_{b\to\infty}F(b)g(b)=0\). If \(\int_a^\infty f\) exists, then \(\lim_{b\to\infty}F(b)\) exists by the continuity of the indefinite integral at~\(\infty\) (Theorem~6), so \(\lim_{b\to\infty}F(b)g(b)\) exists.

We claim \(\int_a^{\infty}Fg'\) exists. First observe
\[\int_a^{\infty}g'=\lim_{b\to\infty}\int_a^bg'=\lim_{b\to\infty}[g(b)-g(a)]=\lim_{b\to\infty}g(b)-g(a)\]
exists by the continuity of the indefinite integral, the fundamental theorem of calculus, and the limit of~\(g\). Since \(g'\le0\),
\[\int_a^{\infty}\abs{g'}=\int_a^{\infty}-g'=-\int_a^{\infty}g'\]
also exists by linearity of the integral. Write \(\abs{F}\le M\). Then \(\abs{Fg'}\le M\abs{g'}\) and \(\int_a^{\infty}M\abs{g'}=M\int_a^{\infty}\abs{g'}\)~exists, so \(\int_a^{\infty}Fg'\) exists by comparison (Corollary~7).

Finally, by continuity of the indefinite integral twice more,
\[\int_a^{\infty}fg=\lim_{b\to\infty}F(b)g(b)-\int_a^{\infty}Fg'\qedhere\]
\end{proof}
\begin{rmk}
These are just Dirichlet's and Abel's tests for convergence of integrals, which are continuous versions of the corresponding discrete tests for infinite series of numbers (Exercises 4.22 and 4.25) and functions (Exercises 11.36 and 11.37). The proofs are essentially the same in the continuous and discrete cases, except integration by parts replaces summation by parts and absolute integrability replaces absolute convergence.
\end{rmk}
\begin{rmk}
Instead of assuming that \(g\)~is nonnegative and decreasing, we may assume only that \(g\)~is bounded and monotone.
\end{rmk}

\section*{Chapter~15}
\begin{rmk}
The monotone convergence theorem (Theorem~1) shows continuity of the integral \emph{when viewed as a function of the integrand}. Specifically, if \(f_k\upto f\) (or \(f_k\downto f\)), then \(\int_I f=\int_I\lim f_k=\lim\int_I f_k\), when this limit exists. Similarly, the results on ``improper'' integrals (Theorems 13.42~and~14.6) show continuity for the integral \emph{when viewed as a function of the interval of integration}. Specifically, if \(I_k\upto I\), then \(\int_I f=\int_{\lim I_k}f=\lim\int_{I_k}f\), when this limit exists.

The proofs in the book for both types of continuity are similar, and involve combining infinitely many ``approximating'' gauges (in the former case for the functions~\(f_k\), and in the latter case for the intervals~\(I_k\)) to construct a single ``limit'' gauge for~\(f\) over~\(I\).
\end{rmk}

\begin{rmk}
The dominated convergence theorem (Theorem~3) actually shows the \emph{absolute integrability} of~\(f\), by comparison (Corollary~13.33 and Exercise~47).
\end{rmk}

\begin{rmk}
The theme of ``domination'' recurs frequently, for example in:
\begin{itemize}[itemsep=0pt]
\item The comparison test for absolute integrability (Corollary~13.33)
\item The comparison tests for ``improper'' integrability (Corollary~13.43(ii) and Corollary~14.7(ii))
\item The dominated convergence theorem and its many applications
\end{itemize}
\end{rmk}

\begin{rmk}
In the proof of the general Leibniz differentiation rule for integrals (Corollary~9), to see that \(D_1G\)~is continuous, let \(\vec{v_0}=(x_0,y_0,z_0)\) and \(\vec{v}=(x,y,z)\) and observe that
\begin{align*}
\lim_{\vec{v}\to\vec{v_0}}D_1G(\vec{v})&=\lim_{\vec{v}\to\vec{v_0}}\int_y^z D_1f(x,t)\dt\\
	&=\lim_{\vec{v}\to\vec{v_0}}\left[\,\int_y^{y_0}D_1f(x,t)\dt+\int_{y_0}^{z_0}D_1f(x,t)\dt+\int_{z_0}^z D_1f(x,t)\dt\right]\\
	&=\lim_{\vec{v}\to\vec{v_0}}\int_y^{y_0}\!\!\!D_1f(x,t)\dt+\lim_{\vec{v}\to\vec{v_0}}\int_{y_0}^{z_0}\!\!\!D_1f(x,t)\dt+\lim_{\vec{v}\to\vec{v_0}}\int_{z_0}^z\!D_1f(x,t)\dt\\
	&=\lim_{x\to x_0}\int_{y_0}^{z_0}D_1f(x,t)\dt\tag{1}\\
	&=\int_{y_0}^{z_0}D_1f(x_0,t)\dt\tag{2}\\
	&=D_1G(\vec{v_0})
\end{align*}
where (1)~follows from the continuity of indefinite integrals (Theorem~13.27, see above) and (2)~follows from Theorem~5.\qed
\end{rmk}

\noindent We provide an alternative proof of Proposition~47.
\begin{prop}
Let \(f:\R\to\R\) and let \(\{E_i\}\)~be a family of pairwise disjoint subsets of~\(\R\) over each of which \(f\)~is absolutely integrable. Then \(f\)~is absolutely integrable over \(E=\bigunion_{i=1}^{\infty}E_i\) if and only if \(\sum_{i=1}^{\infty}\int_{E_i}\abs{f}<\infty\), in which case
\[\int_E\abs{f}=\sum_{i=1}^{\infty}\int_{E_i}\abs{f}\quad\text{and}\quad\int_E f=\sum_{i=1}^{\infty}\int_{E_i}f\]
\end{prop}
\begin{proof}
We decompose \(f\)~into positive and negative parts (Proposition~13.40(i)).

Since \(f\)~is absolutely integrable over each~\(E_i\), \(\pos{f}\)~and~\(\neg{f}\) are integrable over each~\(E_i\). Let \(S_n=\bigunion_{i=1}^n E_i\), so \(\{S_n\}\)~is increasing and \(E=\bigunion_{i=1}^{\infty}S_i\). By disjoint additivity (Proposition~12(ii)), \(\pos{f}\)~and~\(\neg{f}\) are integrable over each~\(S_n\) with
\[\int_{S_n}\pos{f}=\sum_{i=1}^n\int_{E_i}\pos{f}\quad\text{and}\quad\int_{S_n}\neg{f}=\sum_{i=1}^n\int_{E_i}\neg{f}\]
By monotone convergence (Proposition~16), \(\pos{f}\)~and~\(\neg{f}\) are integrable over~\(E\), and equivalently \(f\)~is absolutely integrable over~\(E\), if and only if
\[\lim_{n\to\infty}\int_{S_n}\pos{f}=\sum_{i=1}^{\infty}\int_{E_i}\pos{f}<\infty\quad\text{and}\quad\sum_{i=1}^{\infty}\int_{E_i}\neg{f}<\infty\tag{1}\]
in which case
\[\int_E\pos{f}=\sum_{i=1}^{\infty}\int_{E_i}\pos{f}\quad\text{and}\quad\int_E\neg{f}=\sum_{i=1}^{\infty}\int_{E_i}\neg{f}\]
Now if (1)~holds, then
\[\sum_{i=1}^{\infty}\int_{E_i}\abs{f}=\sum_{i=1}^{\infty}\int_{E_i}\left[\pos{f}+\neg{f}\right]=\int_E\pos{f}+\int_E\neg{f}=\int_E\abs{f}<\infty\]
and similarly
\[\sum_{i=1}^{\infty}\int_{E_i}f=\sum_{i=1}^{\infty}\int_{E_i}\left[\pos{f}-\neg{f}\right]=\int_E\pos{f}-\int_E\neg{f}=\int_E f\]
Conversely, if \(\sum_{i=1}^{\infty}\int_{E_i}\abs{f}<\infty\), then
\[\sum_{i=1}^n\int_{E_i}\pos{f}\le\sum_{i=1}^n\int_{E_i}\abs{f}\le\sum_{i=1}^{\infty}\int_{E_i}\abs{f}<\infty\]
and similarly for \(\sum_{i=1}^n\int_{E_i}\neg{f}\), so (1)~holds.
\end{proof}
\begin{rmk}
This result gives us \emph{countable additivity} of the integral over arbitrary (disjoint) sets, provided that we have absolute integrability over each of the sets and the sum of these integrals is finite.
\end{rmk}

\begin{exer}[1]
The monotone convergence theorem (Theorem~1) does not hold for the Riemann integral.
\end{exer}
\begin{proof}
Let \(I=[0,1]\), \(J=\Q\sect I=\{r_1,r_2,\ldots\}\), and \(J_k=\{r_1,\ldots,r_k\}\). Each~\(C_{J_k}\) has only finitely many discontinuities and so is R-integrable over~\(I\) with \(\Rint_0^1 C_{J_k}=0\) and hence \(\lim_{k\to\infty}\Rint_0^1 C_{J_k}=0\). Moreover, \(C_{J_k}\upto C_J\). However, \(C_J\)~is not R-integrable. (See also Example~12.4.)
\end{proof}

\begin{exer}[5]
The dominated convergence theorem (Theorem~3) does not hold for the Riemann integral.
\end{exer}
\begin{proof}
By the proof of Exercise~1, since \(\abs{C_{J_k}}\le 1\) and \(\Rint_0^1 1=1\).
\end{proof}

\begin{exer}[6 (Bounded convergence theorem)]
Let \(I\)~be a compact interval and suppose \(f_k,f:I\to\R\) with \(\abs{f_k}\le M\) and \(f_k\)~integrable for all \(k\ge1\). If \(f_k\to f\), then \(f\)~is integrable and \(\int_I f=\lim_{k\to\infty}\int_I f_k\).
\end{exer}
\begin{proof}
By the dominated convergence theorem, since \(\int_I M<\infty\).
\end{proof}
\begin{rmk}
This result does not hold for unbounded intervals~\(I\). For example, consider \(I=[0,\infty)\) and \(f_k(x)=e^{-x/k}\). Then \(\abs{f_k}\le 1\) on~\(I\) and \(\int_0^\infty f_k=k<\infty\) for all \(k\ge 1\). Also \(f_k\to 1\) on~\(I\). However, \(\int_0^\infty 1=\infty\). See also Exercise~32.
\end{rmk}

\begin{exer}[7]
The bounded convergence theorem (Exercise~6) does not hold for the Riemann integral.
\end{exer}
\begin{proof}
By the proof of Exercise~5.
\end{proof}

\begin{exer}[9 (Laplace transform)]
Let \(f:[0,\infty)\to\R\) be continuous and define
\[\Lap\{f\}(x)=\int_0^{\infty}e^{-xt}f(t)\dt\]
If \(f\)~is of exponential order---that is, if there exist \(M>0\) and \(a\in\R\) such that \(\abs{f(t)}\le Me^{at}\) for all \(t>0\)---then \(\Lap\{f\}\)~defines a continuous function for \(x>a\).
\end{exer}
\begin{proof}
Define \(g(x,t)=e^{-xt}f(t)\). For fixed \(x>a\), \(\int_0^c g(x,t)\dt\) exists for all \(c\ge 0\) by continuity of~\(g\) in~\(t\). Also \(\abs{g(x,t)}\le Me^{(a-x)t}\) for \(t>0\) and \(\int_0^{\infty}e^{(a-x)t}\dt<\infty\), so \(\Lap\{f\}(x)=\int_0^{\infty}g(x,t)\dt\) exists by comparison.

For fixed \(t>0\), \(g\)~is continuous in~\(x\), so by the continuity of the integral (Theorem~5), \(\Lap\{f\}\)~is continuous for \(x>a\).
\end{proof}

\begin{exer}[12]
\(\Gamma\left(\frac{1}{2}\right)=\sqrt{\pi}\).
\end{exer}
\begin{proof}
We have
\begin{align*}
\Gamma\left(\tfrac{1}{2}\right)&=\int_0^{\infty}t^{-1/2}e^{-t}\dt&&\\
	&=2\int_0^{\infty}\frac{e^{-t}}{2\sqrt{t}}\dt&&\\
	&=2\int_0^{\infty}e^{-u^2}\du&&\text{by substitution of \(u=\sqrt{t}\)}\\
	&=2\left(\frac{\sqrt{\pi}}{2}\right)=\sqrt{\pi}&&\text{by Example~8}\qedhere
\end{align*}
\end{proof}

\begin{exer}[21]
The monotone additivity property (Proposition~16) does not hold for the Riemann integral.
\end{exer}
\begin{proof}
Let \(f=1\ge 0\). Let \(E=\Q\sect[0,1]=\{r_1,r_2,\ldots\}\) and \(E_k=\{r_1,\ldots,r_k\}\), so \(E_k\upto E\). Then \(\Rint_{E_k}f=0\), so \(\lim_{k\to\infty}\Rint_{E_k}f=0\), but \(f\)~is not R-integrable over~\(E\).
\end{proof}

\begin{exer}[22]
The countable additivity property (Proposition~17) does not hold the Riemann integral.
\end{exer}
\begin{proof}
As in the proof of Exercise~21, except let \(E_k=\{r_k\}\).
\end{proof}

\begin{exer}[23]
The collection \(\Null=\{\,E\subseteq\R\mid\int_E 1=0\,\}\) of null sets is closed under countable unions, countable intersections, differences, and subsets.
\end{exer}
\begin{proof}
First observe that if \(E,F\in\Null\), so \(\int_E 1=0=\int_F 1\), then \(\int_{E\union F}1\) and \(\int_{E\sect F}1\) exist by absolute integrability (Proposition~14) and
\[\int_{E\union F}1+\int_{E\sect F}1=\int_E 1+\int_F 1=0\]
by additivity (Proposition~12). It follows that \(\int_{E\union F}1=0=\int_{E\sect F}1\) by positivity of the integral, so \(E\union F\in\Null\) and \(E\sect F\in\Null\). By induction, \(\Null\)~is closed under finite unions and intersections.

Suppose \(\{E_k\}_{k=1}^{\infty}\subseteq\Null\). Define \(S_n=\bigunion_{k=1}^n E_k\) and \(S=\bigunion_{k=1}^{\infty}E_k\). By the above, \(S_n\in\Null\) for all \(n\ge 1\), so \(\int_{S_n}1=0\) for all \(n\ge 1\). Now \(S_n\upto S\), so by monotone convergence (Proposition~16), \(\int_S 1=0\) and \(S\in\Null\). Similarly if \(T_n=\bigsect_{k=1}^n E_k\) and \(T=\bigsect_{k=1}^{\infty}E_k\), then \(T_n\in\Null\) for all \(n\ge 1\) and \(T_n\downto T\), so \(T\in\Null\). Therefore \(\Null\)~is closed under countable unions and intersections.

Now if \(E,F\in\Null\), then by the above and additivity,
\[\int_{E-F}1=\int_E 1-\int_{E\sect F} 1=0\]
so \(E-F\in\Null\) and \(\Null\)~is closed under differences.

Finally, if \(E\in\Null\) and \(F\subseteq E\), then
\[\int_F 1=\int_E 1-\int_{E-F}1=0\]
so \(F\in\Null\) and \(\Null\)~is closed under subsets.
\end{proof}

\begin{exer}[26]
Let \(f,g:I\to\R\). Suppose \(f\)~is absolutely integrable and there exists a sequence~\(\{s_n\}\) of step functions\footnote{A step function is of the form \(s=\sum_{k=1}^n a_kC_{I_k}\) where \(a_k\in\R\) and \(I_k\)~is an interval for \(1\le k\le n\).} with \(\abs{s_n}\le M\) for all \(n\ge 1\) and \(s_n\to g\). Then \(fg\)~is absolutely integrable.
\end{exer}
\begin{proof}
Observe that \(fs_n\)~is integrable and
\[\abs{fs_n}\le\abs{f}\abs{s_n}\le M\abs{f}\]
which is integrable. Also \(fs_n\to fg\). Therefore \(fg\)~is absolutely integrable by the dominated convergence theorem.
\end{proof}

\begin{exer}[31 (Improved dominated convergence theorem)]
Let \(f,f_k,g:I\to\R\). Suppose \(f_k,g\) are integrable with \(\abs{f_k}\le g\) for all \(k\ge 1\) and \(f_k\to f\). Then
\[\lim_{k\to\infty}\int_I\abs{f_k-f}=0\]
\end{exer}
\begin{proof}
By the dominated convergence theorem, \(f\)~is integrable, hence \(f_k-f\) is integrable, and \(\lim_{k\to\infty}\int_I(f_k-f)=0\). Also
\[\abs{f_k-f}\le\abs{f_k}+\abs{f}\le g+g=2g\]
so \(f_k-f\) is absolutely integrable by comparison. Now \(\abs{f_k-f}\to 0\), so by the dominated convergence theorem again,
\[\lim_{k\to\infty}\int_I\abs{f_k-f}=\int_I\lim_{k\to\infty}\abs{f_k-f}=\int_I 0=0\qedhere\]
\end{proof}

\begin{exer}[32]
Let \(f_n(t)=n/(t^2+n^2)\) for \(t\in\R\). Then \(0\le f_n\le 1\) and \(\int_{-\infty}^{\infty} f_n=\pi\) for all \(n\ge 1\), and \(f_n\to 0\).
\end{exer}
\begin{proof}
For all \(t\in\R\) and \(n\ge 1\),
\[0<\frac{n}{t^2+n^2}\le\frac{n}{n^2}=\frac{1}{n}\le 1\]
Also
\begin{align*}
\int_{-\infty}^{\infty}\frac{n}{t^2+n^2}\dt&=\int_{-\infty}^{\infty}\frac{n^2}{n^2u^2+n^2}\du&&\text{by substitution of \(u=t/n\)}\\
	&=\int_{-\infty}^{\infty}\frac{1}{u^2+1}\du&&\\
	&=\left.\tan^{-1}u\,\right|_{-\infty}^{\infty}&&\\
	&=\frac{\pi}{2}-\left(-\frac{\pi}{2}\right)=\pi&&
\end{align*}
Finally,
\[\lim_{n\to\infty}\frac{n}{t^2+n^2}=\lim_{n\to\infty}\frac{1/n}{t^2(1/n^2)+1}=0\qedhere\]
\end{proof}
\begin{rmk}
In this example,
\[\lim_{n\to\infty}\int_{-\infty}^{\infty} f_n=\pi\ne 0=\int_{-\infty}^{\infty}\lim_{n\to\infty} f_n\]
This does not contradict the dominated convergence theorem because \(\{f_n\}\)~is not dominated by an integrable function over all of~\(\R\). Similarly this does not contradict the bounded convergence theorem because \(\R\)~is unbounded.
\end{rmk}

\begin{exer}[33]
Let \(f,f_k,g:\R\to\R\). Suppose \(f_k,g\) are integrable with \(\abs{f_k}\le g\) for all \(k\ge 1\) and \(f_k\to f\). Let \(F_k\)~be the indefinite integral of~\(f_k\), and \(F\)~that of~\(f\), so \(F_k(x)=\int_0^x f_k\) and \(F(x)=\int_0^x f\). Then \(F_k\to F\) uniformly on~\(\R\).
\end{exer}
\begin{proof}
By the dominated convergence theorem (Exercise~31), \(\int_{\R}\abs{f_k-f}\to 0\). Given \(\epsilon>0\), choose~\(N\) such that \(\int_{\R}\abs{f_k-f}<\epsilon\) for all \(k\ge N\). Then
\[\abs{F_k(x)-F(x)}=\bigabs{\int_0^x(f_k-f)}\le\int_0^x\abs{f_k-f}\le\int_{\R}\abs{f_k-f}<\epsilon\]
for all \(x\in\R\) and \(k\ge N\), hence \(F_k\to F\) uniformly on~\(\R\).
\end{proof}

\begin{exer}[34]
Let \(f_k=C_{[k,k+1/k)}\). Then \(f_k\to 0\) on~\(\R\) and \(\int_{\R}f_k\to 0\), but \(\{f_k\}\)~is not dominated by an integrable function on~\(\R\).
\end{exer}
\begin{proof}
It is clear that \(f_k\to 0\) and \(\int_{\R}f_k=1/k\to 0\). Suppose \(g:\R\to\R\) is integrable and \(f_k\le g\) for all \(k\ge 1\). Then
\[\frac{1}{k}=\int_k^{k+1/k}f_k\le\int_k^{k+1/k}g\]
for all \(k\ge 1\), so
\[\sum_{k=1}^n\frac{1}{k}\le\sum_{k=1}^n\int_k^{k+1/k}g\le\int_{\R}g\]
for all \(n\ge 1\). But then \(\infty=\sum_{k=1}^{\infty}1/k\le\int_{\R}g\), so \(g\)~is not integrable after all.
\end{proof}
\begin{rmk}
In this example,
\[\lim_{k\to\infty}\int_{\R}f_k=0=\int_{\R}\lim_{k\to\infty} f_k\]
This shows that domination is not a necessary condition for interchanging limit and integral.
\end{rmk}

\begin{exer}[40 (Fatou)]
Let \(f,f_k:I\to\R\). Suppose \(f_k\ge 0\) and \(f_k\)~is integrable for all \(k\ge 1\) with \(f_k\to f\) and \(\liminf_{k\to\infty}\int_I f_k<\infty\). Then \(f\)~is integrable and
\[\int_I f\le\liminf_{k\to\infty}\int_I f_k\]
\end{exer}
\begin{proof}
Let \(g_k=\inf_{m\ge k}f_m\). Note \(g_k\)~is well defined since \(f_m\ge 0\) for all \(m\ge 1\). Also \(g_k\upto f\) (in other words, \(f=\liminf_{k\to\infty}f_k\)).

We claim \(g_k\)~is integrable for all \(k\ge 1\). Indeed, let \(g_{k,m}=f_k\fmin\cdots\fmin f_{k+m}\). Then \(g_{k,m}\)~is integrable (Proposition~14) and \(\int_I g_{k,m}\ge 0\) for all \(m\ge 1\). Also \(g_{k,m}\downto g_k\), hence \(\int_I g_{k,m}\downto{}\), so by the monotone convergence theorem (Theorem~1), \(g_k\)~is integrable as claimed.

We have \(\int_I g_k\upto{}\). We claim that \(\{\int_I g_k\}\)~is bounded above. Indeed, for any \(1\le k\le n\) we have \(g_k=\inf_{m\ge k}f_m\le f_n\), so \(\int_I g_k\le\int_I f_n\), and hence
\[\int_I g_k\le\inf_{n\ge k}\int_I f_n\le\sup_{k\ge 1}\inf_{n\ge k}\int_I f_n=\liminf_{k\to\infty}\int_I f_k<\infty\]
It follows from the monotone convergence theorem that \(f\)~is integrable and
\[\int_I f=\lim_{k\to\infty}\int_I g_k\le\liminf_{k\to\infty}\int_I f_k\qedhere\]
\end{proof}
\begin{rmk}
We see from the proof that the conclusion can be stated as
\[\int_I\liminf_{k\to\infty}f_k\le\liminf_{k\to\infty}\int_I f_k\]
so this result allows interchange of limit inferior and integral (with inequality) under certain conditions. The proof is essentially one half of the proof of the dominated convergence theorem (Theorem~3).
\end{rmk}

\begin{exer}[46]
If \(f:\R\to\R\) is absolutely integrable and uniformly continuous, then \(\lim_{\abs{x}\to\infty}f(x)=0\).
\end{exer}
\begin{proof}
If \(\lim_{\abs{x}\to\infty}f(x)\ne 0\), then \(\lim_{x\to\infty}f(x)\ne 0\) or \(\lim_{x\to-\infty}f(x)\ne 0\). Suppose \(\lim_{x\to\infty}f(x)\ne 0\) (the other case is similar). Then there is \(\epsilon>0\)  and a sequence \(x_k\to\infty\) with \(x_{k+1}-x_k\ge 1\) such that \(\abs{f(x_k)}\ge\epsilon\) for all \(k\ge 1\). By the uniform continuity of~\(f\), there is \(0<\delta<1/2\) such that \(\abs{f(x)}\ge\epsilon/2\) whenever \(\abs{x-x_k}<\delta\). Now
\[\int_{x_k-\delta}^{x_k+\delta}\abs{f}\ge\int_{x_k-\delta}^{x_k+\delta}\frac{\epsilon}{2}=\epsilon\delta\]
But since the intervals \([x_k-\delta,x_k+\delta]\) are pairwise disjoint, this means
\[\int_{\R}\abs{f}\ge\sum_{k=1}^{\infty}\int_{x_k-\delta}^{x_k+\delta}\abs{f}\ge\sum_{k=1}^{\infty}\epsilon\delta=\infty\]
by additivity, contradicting that \(f\)~is absolutely integrable.
\end{proof}
\begin{rmk}
This result does not hold for continuous functions. Consider \(f:\R\to\R\) a nonnegative, continuous bump function with bumps of height~\(1\) and area~\(1/k^2\) at all integers \(\pm k\ne0\), and zero in between. Then \(f\)~is absolutely integrable since \(\sum 2/k^2<\infty\), but \(\lim_{\abs{x}\to\infty}f(x)\ne 0\) since \(f(\pm k)=1\) for all integers \(k\ne 0\). Note \(f\)~is not uniformly continuous since the bumps get arbitrarily steep as \(\abs{x}\to\infty\).
\end{rmk}

\section*{Chapter~16}
\begin{exer}[10]
Let \(f(x)=\sin(x)/x\) on \(I=[0,\infty)\). Then \(f\)~is integrable, but \(\pos{f}\)~is not integrable (Example~14.10). Let \(E_n=[2n\pi, 2n\pi+\pi]\) and \(E=\bigunion_{n=0}^{\infty}E_n\). Then \(E\)~is measurable (Proposition~4), but \(f\)~is not integrable over~\(E\) since \(fC_E=\pos{f}\).
\end{exer}
\begin{rmk}
This example shows that the class of integrable functions is not closed under restriction to measurable subsets.
\end{rmk}

\begin{exer}[13]
Let \(E\)~be a nonmeasurable subset of~\(\R\). By definition, there exists a compact interval \(I\subseteq\R\) such that \(E\sect I\)~is not integrable. Let \(f=C_{E\sect I}-C_{\comp{E}\sect I}\). Then \(\abs{f}=C_I\)~is integrable, but \(f\)~is not integrable, because \(C_{E\sect I}=(f+\abs{f})/2\) is not integrable.
\end{exer}
\begin{rmk}
This example shows that absolute integrability of a function~\(f\) is a stronger property than integrability of~\(\abs{f}\).
\end{rmk}

\section*{Chapter~17}
\begin{rmk}
Fubini's theorem for double integrals (Theorem~12) is a continuous version of the corresponding discrete result for double series (Exercise~4.16), with absolute integrability replacing absolute convergence.
\end{rmk}

\begin{exmp}[14]
Let \(f(x,y)=e^{-xy}-2e^{-2xy}\). Then for \(y>0\),
\[g(y)=\int_1^{\infty}f(x,y)\dx=\left[\frac{-e^{-xy}}{y}+\frac{e^{-2xy}}{y}\right]_{x=1}^{\infty}=\frac{e^{-y}-e^{-2y}}{y}\]
We claim \(\int_0^{\infty}g\) exists. First, \(\int_c^1 g\)~exists for all \(0<c\le 1\) by continuity of~\(g\). Also \(g(y)\to 1\) as \(y\to 0\) (by L'Hospital), so \(g\)~is bounded on~\((0,1]\) and \(\int_0^1 g\)~exists by comparison. Note \(g(y)>0\) for \(y>0\) since \(e^z\)~is strictly increasing, so \(\int_0^1g>0\). Integration by parts yields
\[\int_1^{\infty}g=\int_1^{\infty}\frac{1}{y^2}\left(\frac{e^{-2y}}{2}-e^{-y}\right)\dy+C\]
which exists by comparison with~\(\int_1^{\infty}1/y^2\dy\).

Now for \(x\ge 1\),
\[h(x)=\int_0^1 f(x,y)\dy=\frac{e^{-2x}-e^{-x}}{x}\]
Observe that \(\int_1^c h\)~exists for all \(c\ge 1\) by continuity of~\(h\), and \(\abs{h}=g\) on~\([1,\infty)\), so \(\int_1^{\infty}h\)~exists by comparison. Since \(h(x)<0\) for \(x\ge 1\), \(\int_1^{\infty}h<0\). By the above,
\[-\infty<\int_1^{\infty}\int_0^1 f(x,y)\dy\dx<0<\int_0^1\int_1^{\infty}f(x,y)\dx\dy<\infty\]
so these two iterated integrals exist but are not equal.
\end{exmp}

\begin{exer}[6]
\(\int_{\R^2}C_{[0,1]}=0\).
\end{exer}
\begin{proof}
We prove something stronger. Let \(f(x,y)=(1/x)C_{(0,1]}(x)\). We claim that \(\int_{\R^2}f=0\), from which the result follows by comparison (setting \(f(0,0)=1\)). We appeal directly to the definition of the integral.

Let \(\epsilon>0\). For each \(n\ge 0\), let \(S_n\)~be an open rectangle (strip) of area~\(\epsilon/4^{n+1}\) surrounding the interval \(I_n=(1/2^{n+1},1/2^n]\). Define a gauge~\(\gamma\) on~\(\R^2\) as follows: if \(x\in I_n\), then \(\gamma(x)\subseteq S_n\), otherwise \(\gamma(x)\)~is arbitrary. Let \(\D\)~be a \(\gamma\)-fine tagged division of~\(\R^2\). Let \(\D_n=\{\,(x,I)\in\D\mid x\in I_n\,\}\). Then
\[S(f,\D_n)=\sum_{(x,I)\in\D_n}\frac{1}{x}v(I)<2^{n+1}v(S_n)=\frac{\epsilon}{2^{n+1}}\]
Therefore
\[S(f,\D)=\sum_{n=0}^{\infty}S(f,\D_n)\le\sum_{n=0}^{\infty}\frac{\epsilon}{2^{n+1}}=\epsilon\qedhere\]
\end{proof}
\begin{rmk}
This proof shows that, although \(\int_{[0,1]}1/x=\infty\), \(\int_{[0,1]^2}1/x=0\)---that is, although there is infinite \emph{area} under the curve~\(1/x\) over the unit interval in~\(\R\), there is no \emph{volume} under the curve over the unit square in~\(\R^2\).
\end{rmk}

\begin{exer}[9 (Fubini)]
Let \(\alpha,\beta:[a,b]\to\R\) be continuous with \(\alpha\le\beta\) and
\[E=[a,b]\times[\alpha,\beta]=\{\,(x,y)\mid a\le x\le b,\ \alpha(x)\le y\le\beta(x)\,\}\]
If \(f:\R^2\to\R\) is continuous, then \(f\)~is integrable over~\(E\) and
\[\int_E f=\int_a^b\int_{\alpha(x)}^{\beta(x)}f(x,y)\dy\dx\]
\end{exer}
\begin{proof}
Let \(m=\min\alpha\) and \(M=\max\beta\). Then \(m\le\alpha\le\beta\le M\), so \(E\subseteq I=[a,b]\times[m,M]\). Note \(E\)~is compact in~\(\R^2\) and \(f\)~is continuous on~\(E\), so \(fC_E\)~is bounded on~\(I\) and is also the limit of a sequence of step functions on~\(I\) (Lemma~9). By Fubini's theorem for bounded functions on bounded intervals (Theorem~11), \(fC_E\)~is integrable over~\(I\), so \(f\)~is integrable over~\(E\), and
\[\int_E f=\int_a^b\int_m^M fC_E\dy\dx\]
But for fixed \(x\in[a,b]\), \(\int_m^M fC_E\dy=\int_{\alpha(x)}^{\beta(x)}f\dy\) by definition of~\(E\), so
\[\int_E f=\int_a^b\int_{\alpha(x)}^{\beta(x)}f(x,y)\dy\dx\qedhere\]
\end{proof}

\begin{exer}[13]
\[\int_{[0,1]^2}\frac{1}{x+y}=2\log 2\qquad\int_{[0,1]^2}\frac{1}{x^2+y^2}=\infty\]
\end{exer}
\begin{proof}
Let \(f(x,y)=1/(x+y)\). Let \(E_n=[1/n,1]^2\), so \(E_n\subseteq E_{n+1}\) and \(\bigunion_{n=1}^{\infty}E_n=(0,1]^2\). Then \(f\)~is continuous and integrable over~\(E_n\), and by Fubini,
\begin{align*}
\int_{E_n}f&=\int_{1/n}^1\int_{1/n}^1 f(x,y)\dx\dy\\
	&=\int_{1/n}^1\left[\log(y+1)-\log(y+1/n)\right]\dy\\
	&=2\log 2+\frac{2\log 2}{n}-\frac{2\log n}{n}-2\left(\frac{n+1}{n}\right)\log\left(\frac{n+1}{n}\right)\\
	&\to 2\log 2\quad\text{as }n\to\infty
\end{align*}
Since \(f>0\) on~\((0,1]^2\), monotone convergence implies \(f\)~is integrable over~\((0,1]^2\), hence also over~\([0,1]^2\) (see Exercise~6), and \(\int_{[0,1]^2}f=2\log 2\).

Let \(g(x,y)=1/(x^2+y^2)\). For \(x,y>0\), \(0<x^2+y^2<x^2+2xy+y^2=(x+y)^2\), so \(h(x,y)=1/(x+y)^2<g(x,y)\). Now
\begin{align*}
\int_{E_n}h&=\int_{1/n}^1\int_{1/n}^1 h(x,y)\dx\dy\\
	&=\int_{1/n}^1\left(\frac{1}{y+1/n}-\frac{1}{y+1}\right)\dy\\
	&=2\log\left(\frac{n+1}{n}\right)-\log\left(\frac{2}{n}\right)-\log 2\\
	&\to\infty\quad\text{as }n\to\infty
\end{align*}
By comparison, \(\int_{E_n}g\to\infty\) as \(n\to\infty\), so \(g\)~is not integrable over~\([0,1]^2\).
\end{proof}

\begin{exer}[16 (Generalized dominated convergence theorem)]
Let \(f_k,f,g:\R^p\to\R\). Suppose \(f_k,g\)~are integrable with \(\abs{f_k}\le g\) almost everywhere for each \(k\ge1\), and \(f_k\to f\) pointwise almost everywhere. Then \(f\)~is integrable and
\[\int_{\R^p}f=\lim_{k\to\infty}\int_{\R^p}f_k\]
\end{exer}
\begin{proof}
Let  \(E_0=\{\,x\in\R^p\mid\lim_{k\to\infty}f_k(x)\ne f(x)\,\}\) and \(E_k=\{\,x\in\R^p\mid\abs{f_k(x)}>g(x)\,\}\), so \(E_k\)~is null for all \(k\ge 0\) and \(E=\bigunion_{k=0}^{\infty}E_k\) is also null. Define \(\overline{f_k}=f_k C_{\comp{E}}\), \(\overline{f}=f C_{\comp{E}}\), and \(\overline{g}=g C_{\comp{E}}\). Since \(\overline{f_k}=f_k\) and \(\overline{g}=g\) almost everywhere, \(\overline{f_k}\)~and~\(\overline{g}\) are integrable and \(\int_{\R^p}\overline{f_k}=\int_{\R^p}f_k\) (Lemma~9). Also \(\abs{\overline{f_k}}\le\overline{g}\) and \(\overline{f_k}\to\overline{f}\). By the dominated convergence theorem, \(\overline{f}\)~is integrable and
\[\int_{\R^p}\overline{f}=\lim_{k\to\infty}\int_{\R^p}\overline{f_k}=\lim_{k\to\infty}\int_{\R^p}f_k\]
But \(f=\overline{f}\) almost everywhere, so \(f\)~is integrable and \(\int_{\R^p}f=\int_{\R^p}\overline{f}\).
\end{proof}

\begin{exer}[18]
Let \(f:G\to\R\) and \(g:H\to\R\) be absolutely integrable limits of step functions and \(h(x,y)=f(x)g(y)\). Then \(h\)~is absolutely integrable over \(I=G\times H\) and \(\int_I h=\int_G f\cdot\int_H g\).
\end{exer}
\begin{proof}
Note \(h\)~is the limit of step functions on~\(\R^2\) and
\begin{align*}
\int_H\int_G\abs{h(x,y)}\dx\dy&=\int_H\int_G\abs{f(x)}\abs{g(y)}\dx\dy\\
	&=\int_H\abs{g(y)}\int_G\abs{f(x)}\dx\dy\\
	&=\int_G\abs{f(x)}\dx\int_H\abs{g(y)}\dy
\end{align*}
which exists by assumption. By Fubini (Theorem~12), \(h\)~is absolutely integrable over~\(I\), and by a similar computation, \(\int_I h=\int_G f\cdot\int_H g\).
\end{proof}

\begin{exer}[24]
\(I=\int_0^{\infty}e^{-x^2}\dx=\sqrt{\pi}/2\).
\end{exer}
\begin{proof}
Note \(I\)~exists by comparison with~\(e^{-x}\) and \(I\ge 0\) since \(e^{-x^2}\ge 0\). Now
\begin{align*}
I^2&=\int_0^{\infty}e^{-x^2}\dx\int_0^{\infty}e^{-y^2}\dy&&\\
	&=\int_0^{\infty}\int_0^{\infty}e^{-(x^2+y^2)}\dx\dy&&\text{by Exercise~18}\\
	&=\int_0^{\pi/2}\int_0^{\infty}e^{-r^2}r\d{r}\d{\,\theta}&&\text{by polar substitution}\\
	&=\int_0^{\pi/2}\frac{1}{2}\d{\,\theta}=\frac{\pi}{4}&&
\end{align*}
It follows that \(I=\sqrt{\pi}/2\).
\end{proof}

\section*{Chapter~18}
\begin{rmk}
Convolution is commutative (Exercise~1) and associative (Exercise~2), and if \(\varphi_n\)~is a delta sequence, then it is an ``approximate identity'' element since \(f\conv\varphi_n\approx f\) under appropriate conditions (Theorem~7).
\end{rmk}

\begin{exer}[3]
Let \(f,g:\R\to\R\). Suppose \(f\)~is bounded and continuous and \(g\)~is continuous and absolutely integrable. Then \(f\conv g\)~exists and is bounded and continuous, with \(\sup\abs{f\conv g}\le\sup\abs{f}\int_{-\infty}^{\infty}\abs{g}\).
\end{exer}
\begin{proof}
Let \(h(x,y)=f(x-y)g(y)\). For \(x\in\R\), \(\int_a^b h(x,y)\dy\) exists for all \(a,b\in\R\) by continuity of \(f\)~and~\(g\). Let \(M=\sup\abs{f}\). Then \(\abs{h(x,y)}\le M\abs{g(y)}\) for all \(y\in\R\) and \(M\abs{g}\)~is integrable, so \(f\conv g(x)=\int_{-\infty}^{\infty}h(x,y)\dy\) exists by comparison. Similarly \(\int_{-\infty}^{\infty}\abs{h(x,y)}\dy\) exists, so \(\abs{f\conv g(x)}\le M\int_{-\infty}^{\infty}\abs{g}\). Since \(x\)~was arbitrary, it follows that \(\sup{\abs{f\conv g}}\le M\int_{-\infty}^{\infty}\abs{g}\). Finally, since \(h\)~is also continuous in~\(x\) for each \(y\in\R\), it follows that \(f\conv g\)~is continuous.
\end{proof}
\begin{rmk}
Contrary to what the book asks us to prove, \(f\conv g\)~is not necessarily integrable. For example, if \(f(x)=1\) and \(g(y)=e^{-y^2}\), then \(f\conv g(x)=\sqrt{\pi}\), which is not integrable.
\end{rmk}

\section*{Chapter~22}
\begin{exer}[11]
A normed linear space \((X,\norm{\cdot})\) is complete if and only if, when \(\{x_k\}\subseteq X\) and \(\sum\norm{x_k}\)~converges, then \(\sum x_k\)~converges.\footnote{See also Proposition~6.5 in~\cite{royden}.}
\end{exer}
\begin{proof}
Suppose \(X\)~is complete and \(\{x_k\}\subseteq X\) with \(\sum\norm{x_k}<\infty\). By the Cauchy criterion for series in~\(\R\) (Proposition~4.5), for any \(\epsilon>0\) there exists~\(N\) such that for all \(n\ge m\ge N\),
\[\bignorm{\sum_{k=m}^n x_k}\le\sum_{k=m}^n\norm{x_k}<\epsilon\]
Therefore \(\sum x_k\)~is Cauchy in~\(X\), and hence convergent by completeness of~\(X\).

Conversely, suppose every absolutely convergent series in~\(X\) converges and \(\{x_k\}\subseteq X\) is Cauchy. We claim \(\{x_k\}\)~has a convergent subsequence \(\{x_{k_j}\}\). Indeed, choose \(x_{k_j}\) such that \(\norm{x_{k_j}-x_{k_{j-1}}}<2^{-j+1}\) for \(j>1\). Set \(s_1=x_{k_1}\) and \(s_j=x_{k_j}-x_{k_{j-1}}\) for \(j>1\). Then
\[\sum_{j=1}^{\infty}\norm{s_j}\le\norm{x_{k_1}}+\sum_{j=2}^{\infty}2^{-j+1}=\norm{x_{k_j}}+1\]
so \(\sum s_j\)~is absolutely convergent and hence convergent. But \(\sum_{i=1}^j s_i=x_{k_j}\), so \(\{x_{k_j}\}\)~converges. It follows that \(\{x_k\}\)~also converges, to the same value, since \(\{x_k\}\)~is Cauchy. Therefore \(X\)~is complete.
\end{proof}

\begin{rmk}
This result helps us better understand the proof of the Riesz-Fischer theorem on the completeness of~\(\L^1\) (Theorem~7). In light of this result, to prove that \(\L^1\)~is complete it suffices to show that every absolutely convergent series of functions in~\(\L^1\) converges in~\(\L^1\).

In the proof in the book, the series is \(f=\sum_{k=1}^{\infty}(f_{n_{k+1}}-f_{n_k})\). We see that \(f\in\L^1\), and that the partial sums converge to~\(f\) under~\(\norm{\cdot}_1\), by comparison with \(g=\sum_{k=1}^{\infty}\abs{f_{n_{k+1}}-f_{n_k}}\) and an appeal to the dominated convergence theorem. We see that \(g\in\L^1\) by an appeal to the monotone convergence theorem.
\end{rmk}

\section*{Chapter~26}
\begin{rmk}
The Arzela-Ascoli theorem (Theorem~11) shows that any bounded, equicontinuous family of (real-valued) functions on a compact metric space is totally bounded. Indeed, if \(\F\)~is such a family, then \(\closure{\F}\)~is closed, bounded, and equicontinuous (Exercise~2), hence compact by Arzela-Ascoli, and hence totally bounded (Theorem~25.6).
\end{rmk}

\begin{exer}[6]
The integral operator \(T:\C[a,b]\to\C[a,b]\) given by \(T(f)(x)=\int_a^x f\) is compact.
\end{exer}
\begin{proof}
Note that \(T(f)\in\C[a,b]\) for \(f\in\C[a,b]\) by integrability of continuous functions and continuity of indefinite integrals. Also \(T\)~is linear by linearity of the integral, and \(T\)~is continuous since \(\supnorm{T(f)}\le\supnorm{f}(b-a)\).

Let \(E\subseteq\C[a,b]\) be bounded. We know \(T(E)\)~is bounded by continuity of~\(T\). We claim \(T(E)\)~is equicontinuous. It then follows that \(\closure{T(E)}\)~is closed, bounded, and equicontinuous (Exercise~2), so \(\closure{T(E)}\)~is compact by Arzela-Ascoli, and \(T\)~is compact.

Let \(\epsilon>0\). Let \(\supnorm{f}\le M\) for all \(f\in E\) and set \(\delta=\epsilon/M\). Then for all \(f\in E\) and \(a\le x\le y\le b\) with \(y-x<\delta\),
\[\abs{T(f)(y)-T(f)(x)}=\bigabs{\int_x^yf}\le\int_x^y\abs{f}\le M(y-x)<M\delta=\epsilon\]
Therefore \(T(E)\)~is equicontinuous as claimed.
\end{proof}

\begin{exer}[12]
Let \(X\)~be a metric space with \(D\subseteq X\) dense and suppose \(f_k:X\to\R\) is continuous. If \(\{f_k\}\)~converges uniformly on~\(D\), \(\{f_k\}\)~converges uniformly on~\(X\).
\end{exer}
\begin{proof}
We first claim \(\{f_i\}\)~converges pointwise on~\(X\). Let \(x\in X\). By density of~\(D\) in~\(X\), \(x=\lim_j x_j\) for some \(\{x_j\}\subseteq D\). Let \(a_{ij}=f_i(x_j)\). Then \(\lim_j a_{ij}=f_i(x)\) exists for all~\(i\) by continuity of the~\(f_i\) on~\(X\). Also \(\lim_i a_{ij}\) exists uniformly for all~\(j\) by uniform convergence of the~\(f_i\) on~\(D\). Therefore \(\lim_i f_i(x)\)~exists as claimed, and \(\lim_i f_i(x)=\lim_j\lim_i f_i(x_j)\) (Proposition~11.5).

Let \(f(x)=\lim_i f_i(x)\) for all \(x\in X\). We claim \(f_i\to f\) uniformly on~\(X\). Let \(\epsilon>0\). Choose~\(N\) such that \(\abs{f(x)-f_i(x)}<\epsilon\) for all \(i\ge N\) and \(x\in D\). For \(x\in X\), write \(x=\lim_j x_j\) with \(\{x_j\}\subseteq D\). Then
\[\abs{f(x_j)-f_i(x_j)}<\epsilon\]
for all \(i\ge N\) and \(j\ge 1\). By the above, \(f(x)=\lim_j f(x_j)\), so letting \(j\to\infty\) yields
\[\abs{f(x)-f_i(x)}\le\epsilon\]
for all \(i\ge N\). Therefore \(f_i\to f\) uniformly as claimed.
\end{proof}
\begin{rmk}
The limit~\(f\) is continuous (Proposition~2).
\end{rmk}
\begin{rmk}
It is an immediate corollary that if \(\sum_{k=1}^{\infty}f_k\)~converges uniformly on~\(D\), then \(\sum_{k=1}^{\infty}f_k\)~converges uniformly on~\(X\).
\end{rmk}

\section*{Chapter~28}
\begin{rmk}
In the proof of Lemma~3, in order to construct~\(f\) intuitively we want to ``normalize''~\(g\) at \(s\)~and~\(t\) by dividing by \(g(s)\)~and~\(g(t)\), and then scale the result by \(a\)~and~\(b\) to yield those values at \(s\)~and~\(t\), respectively. Since \(g(s)\ne g(t)\), we can divide by \(g(s)-g(t)\ne0\), so
\[f(x)=\frac{h(x)}{g(s)-g(t)}\]
To ensure that \(f(s)=a\), we must have \(h(s)=a[g(s)-g(t)]\), which suggests that \(h(x)\)~should contain the term \(a[g(x)-g(t)]\). Similarly to ensure that \(f(t)=b\), we must have \(h(t)=b[g(s)-g(t)]\), which suggests that \(h(x)\)~should contain the term \(b[g(s)-g(x)]\). Adding these terms together, we get
\[f(x)=\frac{a[g(x)-g(t)]-b[g(x)-g(s)]}{g(s)-g(t)}\]
which satisfies \(f(s)=a\) and \(f(t)=b\) as desired.
\end{rmk}

\section*{Chapter~29}
\begin{rmk}
We see that it is best to view the derivative of a function \(f:X\to Y\) between normed linear spaces \(X\)~and~\(Y\) as a continuous linear transformation from~\(X\) to~\(Y\) (Definition~1). The following are special cases:
\begin{itemize}
\item If \(X=Y=\R\), such transformations are uniquely represented by numbers in~\(\R\) under the scalar product, so we identify derivatives with numbers (Definition~9.1).
\item If \(X=\R^n\) and \(Y=\R\), such transformations are uniquely represented by (gradient) vectors in~\(\R^n\) under the dot product, so we identify derivatives with vectors (Definitions 10.4~and~10.5 and Propositions 10.3~and~10.6).
\item If \(X=\R\) and \(Y=\R^m\), such transformations are uniquely represented by vectors in~\(\R^m\) under the scalar product, so we identify derivatives with vectors (Definition~10.10). This is also true if \(Y\)~is an arbitrary normed linear space (Example~5).
\item If \(X=\R^n\) and \(Y=\R^m\), such transformations are uniquely represented by (Jacobian) \(m\times n\) matrices over~\(\R\) under the matrix product, so we can identify derivatives with matrices.
\end{itemize}
Dieudonn\'e calls these identifications ``slavish subservience to the shibboleth of numerical interpretation'' (\cite{dieudonne}, p.~141).
\end{rmk}

\begin{rmk}
We see that many results about differentiable functions on~\(\R\) can be extended to differentiable functions \(f:X\to Y\) on more general normed linear spaces~\(X\) by constructing differentiable curves \(\gamma:[a,b]\to X\) and applying the original results (via the chain rule) to the composites~\(f\after\gamma\). For example, see the proof of the mean value theorem on~\(\R^n\) (Theorem~10.11).
\end{rmk}

\begin{rmk}
A function \(f:X\to Y\) between the normed linear spaces \(X\)~and~\(Y\) is differentiable at~\(x_0\) if and only if there is \(T\in L(X,Y)\) such that the function
\[\delta(x)=\begin{cases}
\norm{f(x)-f(x_0)-T(x-x_0)}/\norm{x-x_0}&\text{if }x\ne x_0\\
0&\text{if }x=x_0
\end{cases}\]
is continuous at~\(x_0\), in which case \(T=\df(x_0)\). We call~\(\delta\) the \emph{differential quotient (d.q.)} for~\(f\) at~\(x_0\).
\end{rmk}

\noindent We provide an alternative proof of the chain rule using the previous remark:
\begin{thm}[Chain rule]
Let \(X\), \(Y\), and~\(Z\) be normed linear spaces. Let \(f:X\to Y\), \(g:Y\to Z\), and \(h=g\after f:X\to Z\). If \(f\)~is differentiable at~\(x_0\) and \(g\)~is differentiable at \(y_0=f(x_0)\), then \(h\)~is differentiable at~\(x_0\) and \(\dh(x_0)=\dg(y_0)\df(x_0)\).
\end{thm}
\begin{proof}
Let \(T=\df(x_0)\), \(U=\dg(y_0)\), and \(V=UT\). Let \(\delta_f\)~be the d.q. for~\(f\) at~\(x_0\) and \(\delta_g\)~the d.q. for~\(g\) at~\(y_0\). We claim
\[\delta_h(x)=\begin{cases}
\norm{h(x)-h(x_0)-V(x-x_0)}/\norm{x-x_0}&\text{if }x\ne x_0\\
0&\text{if }x=x_0
\end{cases}\]
is continuous at~\(x_0\), from which the theorem follows. Observe
\begin{multline*}
h(x)-h(x_0)-V(x-x_0)=g(f(x))-g(f(x_0))-U(f(x)-f(x_0))\\
	+U(f(x)-f(x_0)-T(x-x_0))
\end{multline*}
For \(x\)~close but not equal to~\(x_0\), \(\norm{f(x)-f(x_0)}/\norm{x-x_0}\le\norm{T}+1\), so
\[\delta_h(x)\le(\norm{T}+1)\delta_g(f(x))+\norm{U}\delta_f(x)\]
Now \(\delta_g\after f\)~and~\(\delta_f\) continuously vanish at~\(x_0\), so \(\delta_h\)~does also.
\end{proof}

\begin{rmk}
If a function~\(f\) is differentiable at a point~\(x_0\), then change in~\(f\) \emph{at~\(x_0\)} is well approximated by~\(\df(x_0)\). If \(f\)~is \emph{continuously} differentiable at~\(x_0\), then change in~\(f\) \emph{between any two points sufficiently close to~\(x_0\)} is well approximated by~\(\df(x_0)\) (Corollary~15).
\end{rmk}

\begin{rmk}
The proof of the continuity of~\(\df\) in Proposition~17 is flawed because (assuming \(\df=d_1f=d_2f=0\) outside of~\(D\)) we have
\begin{align*}
\df:X\times Y&\to L(X\times Y,Z)\\
d_1f\after I_1:X&\to L(X,Z)\\
d_2f\after I_2:Y&\to L(Y,Z)
\end{align*}
so the functional equation \(\df=d_1f\after I_1+d_2f\after I_2\) makes no sense.

By Proposition~16,
\[\df(x,y)(h,k)=d_1f(x,y)(h)+d_2f(x,y)(k)\]
for all \((x,y),(h,k)\in X\times Y\), so
\[\df(x,y)=d_1f(x,y)\after P_1+d_2f(x,y)\after P_2\]
for all \((x,y)\in X\times Y\), where \(P_1\in L(X\times Y,X)\) and \(P_2\in L(X\times Y,Y)\) are just the projections \(P_1(h,k)=h\) and \(P_2(h,k)=k\). Define \(\P_1:L(X,Z)\to L(X\times Y,Z)\) by \(\P_1(\varphi)=\varphi\after P_1\) and \(\P_2:L(Y,Z)\to L(X\times Y,Z)\) by \(\P_2(\psi)=\psi\after P_2\). Then
\[\df=\P_1\after d_1f+\P_2\after d_2f\tag{1}\]
Now \(\P_1\)~is linear and
\[\norm{\P_1(\varphi)}=\norm{\varphi\after P_1}\le\norm{\varphi}\norm{P_1}=\norm{\varphi}\]
so \(\P_1\)~is also continuous, and similarly \(\P_2\)~is linear and continuous. It follows from~(1) that \(\df\)~is continuous. For the converse, see Exercise~15 below.
\end{rmk}

\begin{exer}[15]
If \(f:D\subseteq X\times Y\to Z\) is continuously differentiable on~\(D\), then \(d_1f\)~and~\(d_2f\) exist and are continuous on~\(D\).
\end{exer}
\begin{proof}
By Proposition~16, \(d_1f\)~and~\(d_2f\) exist and
\[\df(x,y)(h,k)=d_1f(x,y)(h)+d_2f(x,y)(k)\]
In particular,
\[\df(x,y)(h,0)=d_1f(x,y)(h)\qquad\text{and}\qquad\df(x,y)(0,k)=d_2f(x,y)(k)\]
Letting \(I_1\in L(X,X\times Y)\) and \(I_2\in L(Y,X\times Y)\) be the injections \(I_1(h)=(h,0)\) and \(I_2(k)=(0,k)\), we have
\[\df(x,y)\after I_1=d_1f(x,y)\qquad\text{and}\qquad\df(x,y)\after I_2=d_2f(x,y)\]
Define \(\I_1:L(X\times Y,Z)\to L(X,Z)\) by \(\I_1(\varphi)=\varphi\after I_1\) and \(\I_2:L(X\times Y,Z)\to L(Y,Z)\) by \(\I_2(\psi)=\psi\after I_2\). Then \(\I_1\)~and~\(\I_2\) are linear and continuous and (assuming \(\df=d_1f=d_2f=0\) outside of~\(D\))
\[\I_1\after\df=d_1f\qquad\text{and}\qquad\I_2\after\df=d_2f\]
so \(d_1f\)~and~\(d_2f\) are continuous.
\end{proof}

\section*{Chapter~30}
\begin{rmk}
In the proof of the inverse function theorem (Theorem~1), after the initial reductions have been made so that \(x_0=y_0=0\), \(f:X\to X\), and \(\df(0)=I\), we view~\(f\) as a local ``perturbation of the identity'' near the origin, \(f=I+p\), so that a local inverse~\(g\) of~\(f\) satisfies \(g=I-p\after g\) and is therefore a fixed point of the transformation \(\T:\varphi\mapsto I-p\after\varphi\) on an appropriate function space.

In seeking a function space in which we can apply the Banach fixed point theorem to~\(\T\), we are naturally led to consider closed subspaces of bounded continuous functions on~\(X\), since such spaces are complete. We know about the local behavior of~\(p\) near the origin. More specifically, since \(f\)~is continuously differentiable at the origin, for any \(\epsilon>0\) there exists \(\delta>0\) such that
\[\norm{p(x_1)-p(x_2)}\le\epsilon\norm{x_1-x_2}\qquad\text{for }\norm{x_1},\norm{x_2}\le\delta\]
and, taking \(x_2=0\), \(\norm{p(x_1)}\le\epsilon\norm{x_1}\) for \(\norm{x_1}\le\delta\) (Corollary~29.15). Therefore we restrict our attention to subspaces of functions defined near the origin. If \(\varphi\)~is such a function, then
\[\norm{\T(\varphi)(x)}\le\norm{x}+\norm{p(\varphi(x))}\]
so if \(\norm{x}\le\delta/2\) and \(\norm{\varphi(x)}\le\delta\), then \(\norm{\T(\varphi)(x)}\le(1/2+\epsilon)\delta\) by the above, and \(\norm{\T(\varphi)(x)}<\delta\) if we choose \(\epsilon<1/2\), say \(\epsilon=1/3\). This suggests the subspace
\[S=\{\,\varphi\in\BC(\,B(0,\delta/2),X)\mid\norm{\varphi}\le\delta\,\}\]
By the above, \(S\)~is closed under~\(\T\), and \(S\)~is contracted by~\(\T\) since
\[\norm{\T(\varphi)(x)-\T(\psi)(x)}=\norm{p(\varphi(x))-p(\psi(x))}\le\epsilon\norm{\varphi(x)-\psi(x)}\]
Therefore \(\T\)~has a unique fixed point~\(g\), which is the desired local inverse of~\(f\). We see that \(g\)~has nice properties: it is differentiable, and continuously so if \(f\)~is, with derivative equal to the inverse of the derivative of~\(f\).
\end{rmk}

\begin{rmk}
Recall that an indexed family \(\F=\{\,S_x\mid x\in X\,\}\) of sets is equivalently represented by the indexing function \(\pi:\coprod_{x\in X}S_x\to X\) where
\[\coprod_{x\in X}S_x=\bigunion_{x\in X}(\,\{x\}\times S_x)=\{\,(x,s)\mid x\in X\land s\in S_x\,\}\]
represents the disjoint union over~\(\F\) and \(\pi(x,s)=x\). Indeed, for \(x\in X\), \(S_x\)~is obtained by projecting~\(\pi^{-1}(x)\). Moreover, if \(S_x\ne\emptyset\) for all \(x\in X\), then a choice function for~\(\F\) is obtained by projecting a right inverse of~\(\pi\).

In the proof of the implicit function theorem (Theorem~4), this idea is used. The function \(F(x,y)=(x,f(x,y))\) is an indexing function, where \(F^{-1}(x,0)\)~is the set of all~\(y\) such that \(f(x,y)=0\). The local inverse function for~\(F\) is therefore a local choice function for the indexed family, and yields the implicit function~\(\varphi\) such that \(f(x,\varphi(x))=0\). Since the inverse function has the nice property of being continuously differentiable, so does~\(\varphi\).
\end{rmk}

% References
\begin{thebibliography}{0}
\bibitem{depree} DePree, J.~D. and C.~W. Swartz. \textit{Introduction to Real Analysis.} Wiley, 1988.
\bibitem{dieudonne} Dieudonn\'e, J. \textit{Foundations of Modern Analysis.} Academic Press, 1960.
\bibitem{royden} Royden, H.~L. \textit{Real Analysis}, 3rd~ed. Prentice Hall, 1988.
\end{thebibliography}
\end{document}
